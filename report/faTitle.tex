% !TeX root=main.tex

\university{علم و صنعت ایران}

\faculty{دانشکده مهندسی کامپیوتر}

\department{گروه هوش مصنوعی}

\subject{مهندسی کامپیوتر}

\field{هوش مصنوعی}

\title{نمونه‌سازی تقابلی با استفاده از شبکه‌های مولد تقابلی}

\firstsupervisor{دکتر ناصر مزینی}

\name{یگانه}

\surname{مرشدزاده}

\studentID{۹۶۵۲۱۴۸۸}

\thesisdate{تابستان ۱۴۰۰}

\projectLabel{گزارش پروژه پایانی}

\firstPage
\besmPage
\davaranPage

\vspace{.5cm}

%\renewcommand{\arraystretch}{1.2}
\begin{center}
	\begin{tabular}{| p{8mm} | p{18mm} | p{.17\textwidth} |p{14mm}|p{.2\textwidth}|c|}
		\hline
		ردیف	& سمت & نام و نام خانوادگی & مرتبه \newline دانشگاهی &	دانشگاه یا مؤسسه &	امضـــــــــــــا\\
		\hline
		۱  &	استاد راهنما & دکتر \newline  ناصر مزینی & دانشیار & دانشگاه \newline علم و صنعت ایران &  \\
		\hline
	\end{tabular}
\end{center}


\keywords{
فریب شبكه‌ عصبی مصنوعی،روش نشانه‌ی گرادیان سریع، نمونه تقابلی، حمله جعبه سفید، حمله جعبه سیاه، یادگیری عمیق، شبکه مولد تقابلی
}

\fa-abstract{
مدل‌های از قبل آموزش دیده (در مقیاس بزرگ)، مانند یادگیری عمیق، هم‌اکنون قلب و مرکز اصلی پیشرفت هوش مصنوعی است. با وجود این‌که شبکه‌های مصنوعی پیشرفت و موفقیت چشم‌گیری، در بیشتر وقت‌ها فرای توانایی انسان‌ها، از خود در حل مسئله‌های پیچیده نشان داده است، پژوهش‌های اکنون نشان داده‌اند که این شبکه‌ها نسبت به حمله‌های تقابلی در حالتی که تنها دستکاری‌های کوچکی اعمال شود، به طور کامل شبکه را فریب داده و در نتیجه، این شبکه ها بسیار آسیب پذیر و حساس هستند. 
در این گزارش پس از بیان اهمیت مسئله، یک دسته‌بندی از انواع حملات به شبکه‌های‌عصبی و همچنین مختصری از معمار و نحوه کار کردن شبکه مولد تقابلی بیان شده است. در ادامه به بررسی تاثیر نمونه‌ تقابلی تولید شده توسط روش حمله 
\lr{Adv-GAN}
، که الهام گرفته از شبکه‌های مولد تقابلی است، در فریب شبکه هدف پرداخته شده‌است. 
\\
در انتها نتایج بدست آمده در قالب انواع آمارها و نمودارها آورده شده است که همه نشان دهنده این هستند شبکه هدفی که قبل از حمله دقت 
\lr{۹۹/۳\%}
را داشته است، پس از حمله بسیار موفق توسط بخش مولد شبکه 
\lr{Adv-GAN}
، به دقت 
\lr{۰/۴۳\%}
رسیده است که نشان از میزان موفقیت 
\lr{۹۹/۵۷\%}
حمله
\LTRfootnote{Attack Success Rate}
 دارد.
}

\abstractPage

\newpage\clearpage

