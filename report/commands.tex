% در این فایل، دستورها و تنظیمات مورد نیاز، آورده شده است.
%-------------------------------------------------------------------------------------------------------------------

% در ورژن جدید زی‌پرشین برای تایپ متن‌های ریاضی، این سه بسته، حتماً باید فراخوانی شود
\usepackage{tabularx,ragged2e}
\newcolumntype{B}{>{\Centering}X}
\newcolumntype{S}{>{\Centering\hsize=.4\hsize}X}
\newcolumntype{M}{>{\Centering\hsize=.6\hsize}X}

\usepackage{amsthm,amssymb,amsmath}
% بسته‌ای برای تنطیم حاشیه‌های بالا، پایین، چپ و راست صفحه
\usepackage[top=40mm, bottom=40mm, left=25mm, right=35mm]{geometry}
% بسته‌‌ای برای ظاهر شدن شکل‌ها و تصاویر متن
\usepackage{graphicx}
% بسته‌ای برای رسم کادر
\usepackage{framed} 
% بسته‌‌ای برای چاپ شدن خودکار تعداد صفحات در صفحه «معرفی پایان‌نامه»
\usepackage{lastpage}
% بسته‌ و دستوراتی برای ایجاد لینک‌های رنگی با امکان جهش
\usepackage[pagebackref=false,colorlinks,linkcolor=blue,citecolor=blue]{hyperref}
% چنانچه قصد پرینت گرفتن نوشته خود را دارید، خط بالا را غیرفعال و  از دستور زیر استفاده کنید چون در صورت استفاده از دستور زیر‌‌، 
% لینک‌ها به رنگ سیاه ظاهر خواهند شد که برای پرینت گرفتن، مناسب‌تر است
%\usepackage[pagebackref=false]{hyperref}
% بسته‌ لازم برای تنظیم سربرگ‌ها
\usepackage{fancyhdr}
%
\usepackage{setspace}
\usepackage{algorithm}
\usepackage{algorithmic}
\usepackage{subfigure}
\usepackage[subfigure]{tocloft}

% بسته‌ای برای ظاهر شدن «مراجع» و «نمایه» در فهرست مطالب
\usepackage[nottoc]{tocbibind}
\usepackage[numbers]{natbib}
% دستورات مربوط به ایجاد نمایه
\usepackage{makeidx}
\usepackage{graphicx} 
\makeindex

%%%%%%%%%%%%%%%%%%%%%%%%%%
% دستورات مربوط به اضافه کردن استایل پایتون طبق لینک:
%https://tex.stackexchange.com/questions/583310/insert-inline-code-block-in-latex-document
\usepackage{listings}
\usepackage{xcolor}
\usepackage{realboxes}

\definecolor{gsbggray}     {rgb}{0.90,0.90,0.90} % background gray
\definecolor{gsgray}       {rgb}{0.30,0.30,0.30} % gray
\definecolor{gsgreen}      {rgb}{0.00,0.60,0.00} % green
\definecolor{gsorange}     {rgb}{0.80,0.45,0.12} % orange
\definecolor{gspeach}      {rgb}{1.00,0.90,0.71} % peach
\definecolor{gspearl}      {rgb}{0.94,0.92,0.84} % pearl
\definecolor{gsplum}       {rgb}{0.74,0.46,0.70} % plum
\lstdefinestyle{vpython}{%                       % style for listings
	backgroundcolor=\color{gsbggray},%             % background color
	basicstyle=\footnotesize,%                     % default style
	breakatwhitespace=true%                        % break at whitespace
	breaklines=true,%                              % break long lines
	captionpos=b,%                                 % position caption
	classoffset=1,%                                % STILL DON'T UNDERSTAND THIS
	commentstyle=\color{gsgray},%                  % font for comments
	deletekeywords={print},%                       % delete keywords from the given language
	emph={self,cls,@classmethod,@property},%       % words to emphasize
	emphstyle=\color{gsorange}\itshape,%           % font for emphasis
	escapeinside={(*@}{@*)},%                      % add LaTeX within your code
	frame=tb,%                                     % frame style
	framerule=2.0pt,%                              % frame thickness
	framexleftmargin=5pt,%                         % extra frame left margin
	%identifierstyle=\sffamily,%                    % style for identifiers
	keywordstyle=\sffamily\color{gsplum},%         % color for keywords
	language=Python,%                              % select language
	linewidth=\linewidth,%                         % width of listings
	morekeywords={%                                % VPython/GlowScript specific keywords
		__future__,abs,acos,align,ambient,angle,append,append_to_caption,%
		append_to_title,arange,arrow,asin,astuple,atan,atan2,attach_arrow,%
		attach_trail,autoscale,axis,background,billboard,bind,black,blue,border,%
		bounding_box,box,bumpaxis,bumpmap,bumpmaps,camera,canvas,caption,capture,%
		ceil,center,clear,clear_trail,click,clone,CoffeeScript,coils,color,combin,%
		comp,compound,cone,convex,cos,cross,curve,cyan,cylinder,data,degrees,del,%
		delete,depth,descender,diff_angle,digits,division,dot,draw_complete,%
		ellipsoid,emissive,end_face_color,equals,explog,extrusion,faces,factorial,%
		False,floor,follow,font,format,forward,fov,frame,gcurve,gdisplay,gdots,%
		get_library,get_selected,ghbars,global,GlowScript,graph,graphs,green,gvbars,%
		hat,headlength,headwidth,height,helix,hsv_to_rgb,index,interval,keydown,%
		keyup,label,length,lights,line,linecolor,linewidth,logx,logy,lower_left,%
		lower_right,mag,mag2,magenta,make_trail,marker_color,markers,material,%
		max,min,mouse,mousedown,mousemove,mouseup,newball,norm,normal,objects,%
		offset,one,opacity,orange,origin,path,pause,pi,pixel_to_world,pixels,plot,%
		points,pos,pow,pps,print,print_function,print_options,proj,purple,pyramid,%
		quad,radians,radius,random,rate,ray,read_local_file,readonly,red,redraw,%
		retain,rgb_to_hsv,ring,rotate,round,scene,scroll,shaftwidth,shape,shapes,%
		shininess,show_end_face,show_start_face,sign,sin,size,size_units,sleep,%
		smooth,space,sphere,sqrt,start,start_face_color,stop,tan,text,textpos,%
		texture,textures,thickness,title,trail_color,trail_object,trail_radius,%
		trail_type,triangle,trigger,True,twist,unbind,up,upper_left,upper_right,%
		userpan,userspin,userzoom,vec,vector,vertex,vertical_spacing,visible,%
		visual,vpython,VPython,waitfor,white,width,world,xtitle,yellow,yoffset,%
		ytitle%
	},%
	morekeywords={print,None,TypeError},%          % additional keywords
	morestring=[b]{"""},%                          % treat triple quotes as strings
	numbers=left,%                                 % where to put line numbers
	numbersep=10pt,%                               % how far line numbers are from code
	numberstyle=\bfseries\tiny,%                   % set to 'none' for no line numbers
	showstringspaces=false,%                       % show spaces in strings
	showtabs=false,%                               % show tabs within strings
	stringstyle=\color{gsgreen},%                  % color for strings
	upquote=true,%                                 % how to typeset quotes
}%

\newcommand*{\pythonline}[1]{\Colorbox{gsbggray}{\lstinline[style=vpython]{#1}}}

%%%%%%%%%%%%%%%%%%%%%%%%%%
% فراخوانی بسته زی‌پرشین و تعریف قلم فارسی و انگلیسی
\usepackage[extrafootnotefeatures]{xepersian}
\settextfont[Scale=1]{XB Niloofar.ttf}
\setlatintextfont[Scale=0.9]{Times New Roman}

%%%%%%%%%%%%%%%%%%%%%%%%%%
% چنانچه می‌خواهید اعداد در فرمول‌ها، انگلیسی باشد، خط زیر را غیرفعال کنید
%\setdigitfont[Scale=1]{XB Zar.ttf}%{Persian Modern}
%%%%%%%%%%%%%%%%%%%%%%%%%%
% تعریف قلم‌های فارسی و انگلیسی اضافی برای استفاده در بعضی از قسمت‌های متن
%\defpersianfont\titlefont[Scale=1]{XB Niloofar.ttf}
% \defpersianfont\iranic[Scale=1.1]{XB Zar Oblique}%Italic}%
% \defpersianfont\nastaliq[Scale=1.2]{IranNastaliq}

\setcounter{tocdepth}{3}
\setcounter{secnumdepth}{3}

%%%%%%%%%%%%%%%%%%%%%%%%%%
% دستوری برای حذف کلمه «چکیده»
\renewcommand{\abstractname}{}
% دستوری برای حذف کلمه «abstract»
%\renewcommand{\latinabstract}{}
% دستوری برای تغییر نام کلمه «اثبات» به «برهان»
\renewcommand\proofname{\textbf{برهان}}
% دستوری برای تغییر نام کلمه «کتاب‌نامه» به «مراجع»
\renewcommand{\bibname}{مراجع}
% دستوری برای تعریف واژه‌نامه انگلیسی به فارسی
\newcommand\persiangloss[2]{#1\dotfill\lr{#2}\\}
% دستوری برای تعریف واژه‌نامه فارسی به انگلیسی 
\newcommand\englishgloss[2]{#2\dotfill\lr{#1}\\}
% تعریف دستور جدید «\پ» برای خلاصه‌نویسی جهت نوشتن عبارت «پروژه/پایان‌نامه/رساله»
\newcommand{\پ}{پروژه/پایان‌نامه/رساله }

%\newcommand\BackSlash{\char`\\}

%%%%%%%%%%%%%%%%%%%%%%%%%%
\SepMark{-}

% تعریف و نحوه ظاهر شدن عنوان قضیه‌ها، تعریف‌ها، مثال‌ها و ...
\theoremstyle{definition}
\newtheorem{definition}{تعریف}[section]
\theoremstyle{theorem}
\newtheorem{theorem}[definition]{قضیه}
\newtheorem{lemma}[definition]{لم}
\newtheorem{proposition}[definition]{گزاره}
\newtheorem{corollary}[definition]{نتیجه}
\newtheorem{remark}[definition]{ملاحظه}
\theoremstyle{definition}
\newtheorem{example}[definition]{مثال}

%\renewcommand{\theequation}{\thechapter-\arabic{equation}}
%\def\bibname{مراجع}
\numberwithin{algorithm}{chapter}
\def\listalgorithmname{فهرست الگوریتم‌ها}
\def\listfigurename{فهرست شکل‌ها}
\def\listtablename{فهرست ‌جدول‌ها}

%%%%%%%%%%%%%%%%%%%%%%%%%%%%
% دستورهایی برای سفارشی کردن سربرگ صفحات
% \newcommand{\SetHeader}{
% \csname@twosidetrue\endcsname
% \pagestyle{fancy}
% \fancyhf{} 
% \fancyhead[OL,EL]{\thepage}
% \fancyhead[OR]{\small\rightmark}
% \fancyhead[ER]{\small\leftmark}
% \renewcommand{\chaptermark}[1]{%
% \markboth{\thechapter-\ #1}{}}
% }
%%%%%%%%%%%%5
%\def\MATtextbaseline{1.5}
%\renewcommand{\baselinestretch}{\MATtextbaseline}
\doublespacing
%%%%%%%%%%%%%%%%%%%%%%%%%%%%%
% دستوراتی برای اضافه کردن کلمه «فصل» در فهرست مطالب

\newlength\mylenprt
\newlength\mylenchp
\newlength\mylenapp

\renewcommand\cftpartpresnum{\partname~}
\renewcommand\cftchappresnum{\chaptername~}
\renewcommand\cftchapaftersnum{:}

\settowidth\mylenprt{\cftpartfont\cftpartpresnum\cftpartaftersnum}
\settowidth\mylenchp{\cftchapfont\cftchappresnum\cftchapaftersnum}
\settowidth\mylenapp{\cftchapfont\appendixname~\cftchapaftersnum}
\addtolength\mylenprt{\cftpartnumwidth}
\addtolength\mylenchp{\cftchapnumwidth}
\addtolength\mylenapp{\cftchapnumwidth}

\setlength\cftpartnumwidth{\mylenprt}
\setlength\cftchapnumwidth{\mylenchp}	

\makeatletter
{\def\thebibliography#1{\chapter*{\refname\@mkboth
   {\uppercase{\refname}}{\uppercase{\refname}}}\list
   {[\arabic{enumi}]}{\settowidth\labelwidth{[#1]}
   \rightmargin\labelwidth
   \advance\rightmargin\labelsep
   \advance\rightmargin\bibindent
   \itemindent -\bibindent
   \listparindent \itemindent
   \parsep \z@
   \usecounter{enumi}}
   \def\newblock{}
   \sloppy
   \sfcode`\.=1000\relax}}
\makeatother
